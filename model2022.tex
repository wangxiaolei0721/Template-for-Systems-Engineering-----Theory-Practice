% 在texlive环境中使用xalatex编译
\documentclass[a4paper, c5size, twoside]{ctexart}
\usepackage{amsmath, amsthm, amssymb, appendix, bm, graphicx, hyperref, mathrsfs}
\usepackage{fancyhdr} % 页眉设置包
\usepackage{booktabs} % 表格横线包

%    定义页面边距
\setlength{\textheight}{24.5 true cm} \setlength{\textwidth}{16.5 true cm } % 正文长宽
\setlength{\hoffset}{3pt} \setlength{\voffset}{-5mm} % 页面整体偏移量
\setlength{\oddsidemargin}{0pt} % 额外的左边距
\setlength{\evensidemargin}{0pt} % 额外的左边距
\setlength{\topmargin}{0pt} % 额外的上边距
\setlength{\headsep}{0.3  true cm} % 页眉与正文间距
\setlength{\footskip}{2mm}

%%%%%%%%%%%%%%%%%%%%%%%%%%%%%%%%%%%%%%%%%%%%%%%%%%%%%%%%
\def \Y {{\rm 2022}}  % 出版年
\def \CM {{\rm X}}   % 出版月
\def \EM {{\rm XXX.,}} % 出版月(英文)
\def \Vol {{\rm 39}}  % 卷号
\def \No {{\rm X}}    % 期号

% 首页页眉页脚定义
\fancypagestyle{plain}{ 
\fancyhf{} %plain 小五宋体
\vspace{4mm}
\lhead{\small  第~\Vol~卷 第~\No~期\\
\vspace{1mm}{\Y~年~\CM~月}}
\chead{\centering{ \normalsize 系统工程理论与实践\\
\small Systems Engineering --- Theory \& Practice}}
\rhead{\small {Vol.\Vol ~No.\No}\\ \vspace{1mm}
{\EM ~\Y}} \lfoot{} \cfoot{} \rfoot{}
\renewcommand{\headrule}{%
\vspace{1mm}
\hrule height0.4pt width \headwidth \vskip1.0pt%
\hrule height0.4pt width \headwidth \vskip-2pt}
}
% 首页后根据奇偶页不同设置页眉页脚
% R,C,L分别代表左中右,O, E代表奇偶页
%%%%%%%%%%%%%%%%%%%%%%%%%%%%%%%%%%%%%%%%%%%%%%%%%%%%%%%%%%%%%%%%
\pagestyle{plain} \fancyhf{} \fancyhead[RE]{{\small{  %plain 小五宋体
第~\Vol~卷~~~}}} \fancyhead[CE]{{\small
{系~统~工~程~理~论~与~实~践}}}
\fancyhead[LE,RO]{~~~\small\thepage ~~~}
\fancyhead[CO]{{\small{ 张三等: 题目}}}
\fancyhead[LO]{{\small {~~~第~\No~期}}} \lfoot{}
\cfoot{} \rfoot{}
\renewcommand{\headrule}{%
\hrule height0.4pt width \headwidth \vskip1.0pt%
}

%    定义行距 字体缩写等
\renewcommand{\baselinestretch}{1.2}
\setlength{\parindent}{2em} %段首缩进两字符
\setlength{\parskip}{1pt}
\def\d{\displaystyle} \def\n{\noindent}
\def\ST{\songti\rm\relax}
\def\HT{\heiti\bf\relax}
\def\FS{\fangsong\relax}
\def\KS{\kaishu\relax}
\def\sz{\small \zihao{-5}}
\def\vs{\vspace{0.3cm}}
\def\ay{\arraycolsep=1.5pt}
% 此行使文献引用以上标形式显示
\newcommand{\supercite}[1]{\textsuperscript{\cite{#1}}}

\def\SEC#1#2#3{\vspace*{.2in} \begin{center}
{\LARGE\zihao{3}\HT #1}\\[.2in]
\zihao{4}\fangsong#2\\[.1in]
\small\zihao{-5}#3 \end{center}}

\def\ESEC#1#2#3{\vskip.2in \begin{center}
{\Large \HT #1}\\[.2in]
\normalsize #2\\[.1in]
\footnotesize #3 \end{center}}

\def\SUB#1{\vspace{.15in} \leftline{\large\bf\heiti\zihao{-4}#1}
\vspace{.07in}}
\def\sub#1{\leftline{\bf\heiti\zihao{5}#1}}

\def\REFERENCE{\vspace*{.2in}
{\noindent\bf\heiti\zihao{5}参考文献} \vspace*{.1in}}
\renewcommand\refname{}% 去掉参考文献中自动生成的参考文献标题

%%%%%%%%%%%%%%%%%%%%%%%%%%%%%%%%%%%%%%%%%%%%%%%%%%%%%%%%%%%%%%%%
%  \songti  \heiti  \kaishu \fangsong \lishu \youyuan
%点数(pt) 相应中文字号 控制命令 %  25 一号 \Huge%  20 二号 \huge%  17 三号 \LARGE
%  14 四号 \Large%  12 小四号 \large%  10 五号 \normalsize%  9 小五号 \small
%  8 六号 \footnotesize%  7 小六号 \scriptsize%  5 七号 \tiny
%%%%%%%%%%%%%%%%%%%%%%%%%%%%%%%%%%%%%%%%%%%%%%%%%%%%%%%%%%%%%%%%

%---------------------题目、作者、单位, 中图分类号(必须提供)------------------
\title{
\vspace{-12mm}   \leftline{\small doi:\
10.12011/1000-6788(2019)01-0001-11 \qquad \qquad {中图法分类号:}\ \ XXX %中图分类号从《中国图书馆分类法》(第四版)中查得。
\qquad\qquad{文献标志码:\ \ A}} 
%\SEC{中文题目}{作者}{单位}
\SEC{多因素时变Markov链模型下考虑信用风险的互换期权定价}
{陈正声$^{1,2}$,秦学志$^{2}$} {(1. 大连银行\ 风险管理部, 大连
116001; 2. 大连理工大学\ 管理与经济学部, 大连 116024)}		

%-----------------------脚注--------------------------------------------------
\footnotetext{\hspace {-4.7mm}{\HT 收稿日期:}\ 2017-11-14} 
\footnotetext{\hspace {-4.7mm}{\HT
	作者简介:}\ 孙会霞(1984--),女,山东临沂人,博士研究生,研究方向:
公司金融,资本市场.} 
\footnotetext{\hspace {-4.7mm}{\HT 基金项目:}\
国家自然科学基金(71102118)} 
\footnotetext{\hspace {-4.7mm}{\HT Foundation item:}\
National Natural Science Foundation of China
(71102118)}
\footnotetext{\hspace {-4.7mm}{\HT 中文引用格式:}\
陈正声,秦学志.多因素时变Markov链模型下考虑信用风险的互换期权定价[J].系统工程理论与实践,
2019, 39(X): 1--10.} 
\footnotetext{\hspace {-4.7mm}{\HT 英文引用格式:}\ Chen Z S,
Qin X Z. The valuation of swaption with counterparty risk under a
multi-factor time-varying Markov chain model[J]. Systems
Engineering --- Theory \& Practice, 2019, 39(X): 1--10.}
}
\date{}     % 这一行用来去掉默认的日期显示

\begin{document}
% \setcounter{page}{1} % 此命令为重新设置页码,即把 1 处改为需要的页码即可

\maketitle
%---------------------摘要、关键词------------------
\vspace{-18mm} 
{\narrower\zihao{5}\fangsong \noindent{\heiti 摘\quad
	要} 
\ \


\vskip.05in
\noindent{\heiti 关键词}  \ \


} \ \ 


%%%%%%%%%%%%%%%英文部分%%%%%%%%%%%%%%%%%%%%%%%
%\ESEC{英文题目}{作单}{单位}
\ESEC{The valuation of swaption with counterparty risk under a
multi-factor time-varying Markov chain model} {CHEN
Zhengsheng$^{1,2}$, \ QIN Xuezhi$^{2}$} {(1. Department of Risk
Management, Bank of Dalian, Dalian 116001, China; \\2. Faculty of
Management and Economics, Dalian University of Technology, Dalian
116024, China)}

\vskip.05in 
\rm \noindent {\narrower\small {\bf Abstract}\ \




\vskip.05in
\noindent{\bf Keywords}\ \


}



%%%%%%%%%%%%%%%%%%%%%%%注意事项%%%%%%%%%%%%%%%%%%%%%%%%%%%%%%%%%%
% 全文所有的标点符号(包括括号、冒号、分号、逗号、句号等)都用``英文"中的标点符号, 即``半角"加空格.

%%%%%%%%%%%%%%%%%%%%%%正文%%%%%%%%%%%%%%%%%%%%%%%%%%%%%%%%%

\SUB{1\ \ 引言}%一级标题


%\sub{1.1\ \ }二级标题命令
%{\HT 定理1}\ \  {\HT 证明}\ \

\begin{equation}
f(d_{ij}) = {\rm e}^{-\gamma d_{ij}}.
\end{equation}


\begin{equation}
\begin{aligned}
	B'&=-\partial\times E, \\
	E'&=\partial\times B-4\pi j.
\end{aligned}
\end{equation}


\begin{equation}f(x)=
\begin{cases}
	1,&-1<x<1, \\
	0,&\mbox{其他}.
\end{cases}
\end{equation}


%%%%%%%%%%%%%%%%%%%%插入表格%%%%%%%%%%%%%%%%%%%%%%%%%%%%%%%%%%
%\multirow{2}*{活动名称}%表格行中的内容上下居中.
%\cmidrule(r){3-4}%表格划线中间断开
\vspace{2mm}
\begin{center}{\sz
	{\textbf{表1\ \  不同期限债券利率的描述性统计量}}\\
	\begin{tabular}{lccccccc} \toprule
		& 0.5年期 &  1年期 &  2年期 &  3年期 &  5年期 & 7年期 & 10年期 \\
		\hline
		均值 & 0.0352   &0.0361  & 0.0385 & 0.0402   & 0.0433  & 0.0458  & 0.0475\\
		标准差  & 0.0176   & 0.0166  & 0.0153  &0.0137  & 0.0109  & 0.0097  &0.0076\\
		\bottomrule
\end{tabular}}
\end{center}\vspace{2mm}

%%%%%%%%%%%%%%%%%%%%插入图片%%%%%%%%%%%%%%%%%%%%%%%%%%%%%%%%%%
\vspace{2mm}\begin{center}
%\includegraphics[scale=1]{1r.eps}\\
%{\bf \sz 图1\ \
	%沪深股市指数对数收益率用正态核密度估计方法积分变换后的序列Q-Q图}
\end{center}\vspace{2mm}



\REFERENCE
%%%%%%%%%%%%%%%%%%%%%参考文献%%%%%%%%%%%%%%%%%%%%%%%%%%%%%%%%%%%%%%%%
{\small \baselineskip 12pt
%为反映论文的学术水平和创新程度,请主要引用最近3年以内发表的文献(最好引用期刊类文献).
%作者可在本刊网站: http://www.sysengi.com免费下载查阅创刊以来的所有论文全文.
%参考文献列表必须按正文中引用的先后顺序排序,所有被引用的文献须在正文中标明引用位置:
%不做句子成份的文献以上角标形式标引,直接用$^{[1]}$形式标引即可,做句子成份的文献则与正文平排.
%对于中外文的个人著者,一律采用姓前名后的著录方式.用汉语拼音书写的中国著者姓用全拼,名用缩写;欧美著者的名用缩写字母;缩写名后不用缩写点,
%例如: Zhang J T, Calms R B. 作者数量多于三个人时,在第三个作者后面加``,等" 或``, et al".作者名之间用逗号分隔,不用``and".
\begin{thebibliography}{99} 
	\vspace{-15mm}
	%1
	\bibitem{1}
	孙会霞, 苏峻, 何佳.
	股票供给控制、需求曲线与股价的反应------基于创业板的经验数据[J].
	系统工程理论与实践, 2013, 33(1): 1--11. \\
	Sun H X, Su J, He J. Control on supply of shares, demand curves
	and the reaction of stock price --- Based on the data of
	Chinext[J]. Systems Engineering --- Theory \& Practice, 2013,
	33(1): 1--11.%期刊著录格式
	%2
	\bibitem{2}
	Chen L. Corporate yield spreads and bond liquidity[R]. East Lansing: Michigan State University, 2005. %研究报告著录格式
	%3
	\bibitem{3}
	席酉民, 王亚刚. 和谐社会秩序形成机制的系统分析:
	和谐管理理论的启示和价值[C]//
	中国系统工程学会第十四届学术年会论文集, 2006: 18--29.%会议论文著录格式
	%4
	\bibitem{4}
	Turcotte D L. Fractals and chaos in geology and
	geophysics[M/OL]. New York: Cambridge University Press, 1992
	[1998-09-23]. http://wwwsegorg/reviews/mccorm30.html.%电子文献著录格式,方括号内为引用日期
\end{thebibliography}}
\end{document}